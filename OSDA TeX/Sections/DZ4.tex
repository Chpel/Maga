\section{Home assignment №4, Pchelintsev Ilya}

\subsection{Task 1: LazyFCA}

\subsubsection{Order of the dataset}

\begin{figure}[h]
\centering
\begin{minipage}{0.45\textwidth}
\includegraphics[width=0.8\textwidth]{4_1_Dataset1}
\end{minipage}
\hfill
\begin{minipage}{0.45\textwidth}
\includegraphics[width=0.8\textwidth]{4_1_Dataset2}
\end{minipage}
\caption{train and test subdatasets for tasks 1.1 and 1.2}
\end{figure}

\begin{lstlisting}[language=Python, caption=LazyFCA algorithm]
def LazyFCA(t0, df_train, y_train, cat_feats, num_feats, proof=False, alpha=0.75):
    """
    t0: one object of test dataset
    df_train: train dataset
    y_train: target values of the train dataset
    cat_feats: names of categorical features
    num_feats: names of numerical features
    proof: boolean value of writing the number of object which caused succesfull classification
    alpha: a fraction of object which is enough for classifying
    """
    for _, t_x in df_train.iterrows():
        sim_cfeats = cat_feats[t_x.loc[cat_feats] == t0.loc[cat_feats]]
        num_df0 = pd.DataFrame([t_x, t0])[num_feats]
        min_vals = num_df0.min()
        max_vals = num_df0.max()
        if len(sim_cfeats) > 1:
            cat_sim_mask = (df_train[sim_cfeats] == t0[sim_cfeats]).all(axis=1)
        elif len(sim_cfeats) == 1:
            cat_sim_mask = df_train[sim_cfeats[0]] == t0[sim_cfeats[0]]
        else:
            cat_sim_mask = pd.Series([True] * df_train.shape[0], index=df_train.index)
        extsim = (cat_sim_mask & \
                       (df_train[num_feats] >= min_vals).all(axis=1) & \
                        (df_train[num_feats] <= max_vals).all(axis=1))
        y_extsim = y_train.loc[extsim].value_counts()
        y_extsim /= y_extsim.sum()
        res = y_extsim[y_extsim > 0.75]
        if len(res) > 0:
            if proof:
                print('proof', _)
            return res.index[0]
    return None

res = df_test.apply(lambda x: LazyFCA(x, df_train, y_train, cat_f, num_f, True), axis=1)
print(f'accuracy: {(res == y_test).sum() / res.shape[0]}')
\end{lstlisting}

The result of the LazyFCA algorithm for the test subset was 'FFF' (all were 'proved' by the first object), with the accuracy equal to zero. 
On the other hand, for the randomized dataset (see fig.\ref{fig:data1_2}) the result was the opposite - 'TTT' with the ideal accuracy of 1.0.
It is obvious that the algorithm is highly dependent to the order of the train dataset, which require cross-validation of it.


\subsubsection{Cross-validation}

\begin{figure}[h]
\centering
\includegraphics[width=0.5\textwidth]{4_1_3}
\caption{The code for the task 1.3 and result for each subset}
\end{figure}

\subsubsection{Different intersections}

Consider two new ways of intersections - the first way is the 'minimal' half-interval:

\[ [a_1, \infty) \cap [a_2, \infty) = [min(a_1,a_2), \infty) \]

Considering the train and test subsets same as in task 1.2, the result is the following (fig. \ref{fig:min_int})

\begin{figure}[h]
\centering
\label{fig:min_int}
\includegraphics[width=0.75\textwidth]{4_1_4}
\caption{The code for the task 1.4 (commented part is the changed calculation of exstim-mask in the "LazyFCA" function) and the result}
\end{figure}

The second way is the 'maximal' half-interval:

\[ [a_1, \infty) \cap [a_2, \infty) = [max(a_1,a_2), \infty) \]

Considering the train and test subsets same as in task 1.2, the result is the following (fig. \ref{fig:min_int})

\begin{figure}[h]
\centering
\label{fig:min_int}
\includegraphics[width=0.75\textwidth]{4_1_5}
\caption{The code for the task 1.5 (commented part is the changed calculation of exstim-mask in the "LazyFCA" function) and the result}
\end{figure}


\subsubsection{References}

All code is available in my \href{https://github.com/Chpel/Maga/blob/main/OSDA\%20.ipynb/Untitled.ipynb}{repository}.

\subsection{Task 2: Graphs and hypothesis}

\begin{figure}[h]
\centering
\begin{minipage}{0.49\textwidth}
\includegraphics[width=0.99\textwidth]{Pos_lattice}
\end{minipage}
\hfill
\begin{minipage}{0.49\textwidth}
\includegraphics[width=0.99\textwidth]{Neg_lattice}
\end{minipage}
\caption{Positive and negative lattices of the task 2}
\end{figure}

The minimal positive hypothesis $H^{+}=\{1,2,3,4\}$, as there are no counterparts from the negative dataset.
The minimal negative hypothesis is $H^{-}=\{5,6,7\}$, as there are no counterparts from the negative dataset.
As a result, according to the gathered hypothesises:

\begin{itemize}
\item $G_8 \rightarrow +$, as $H^{+} \in G_8 \& H^{-} \notin G_8$
\item $G_9 \rightarrow -$, as $H^{-} \in G_9 \& H^{+} \notin G_9$
\item $G_{10} \rightarrow +$, as $H^{+} \in G_{10} \& H^{-} \notin G_{10}$
\item $G_{11} \rightarrow +$, as $H^{+} \in G_{11} \& H^{-} \notin G_{11}$
\end{itemize}
