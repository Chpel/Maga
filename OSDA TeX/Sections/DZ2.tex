\section{Home assignment №2, Pchelintsev Ilya}

\subsection{Problem 1: Prove Isomorphism and Determine
Topological Sorting (3 points)}

Given: 2 sets $A$, $B$ and their respective relations $R_1$ and $R^{'}_2$:

\begin{align*}
 A &= \{1,2,3,4,5,6\} &  B &=\{ 'a','b','c','d','e','f' \} \\
 R_1 &= \{ (a,b)\ |\ (a+b) mod\ 2 = 0\} & R^{'}_2 &= \{ (x,y)\ |\ |x-y|\ mod 2\ = 0 \}
\end{align*}

\subsubsection{Proof of isomorphism:}

\begin{enumerate}
\item Both sets have two cluques:
\[ Q_1 = \{1,3,5\},\ \ Q_2 = \{2,4,5\} \]
\[ Q{'}_1 = \{'a', 'c', 'e'\},\ \ Q^{'}_2 = \{'b','d','f'\} \]

\item Consider a bijective function $f: A \leftrightarrow B$: ''letter $\leftrightarrow$ number of the letter in the alphabethical order''
\item Result:
\begin{align*}
1 &\leftrightarrow 'a' & 3 &\leftrightarrow 'c' & 5 &\leftrightarrow 'e' \\
2 &\leftrightarrow 'b' & 4 &\leftrightarrow 'd' & 6 &\leftrightarrow 'f'
\end{align*}
\item The isomorphism between relations is proved, as the considered function bijectively connects entities of respective cliques. 
\end{enumerate}

\subsubsection{Topological sorting}

\begin{enumerate}
\item For $R_1$: $\{1, 3, 5, 2, 4, 6\}$
\item For $R_1$: $\{'a', 'c', 'e', 'b', 'd', 'f'\}$
\end{enumerate}


\subsection{Problem 2: On Posets Properties (3 points)}

Given: a set $S$ with a relation $R$:

\begin{align*}
S &= \{1,2,3,4,6,8,12\} \\
R &= \{ a \leq b\ |\ b mod\ a = 0
\end{align*}

\subsubsection{Hasse diagram}

\begin{figure}[h]
\centering
\includegraphics[width=0.3\textwidth]{2_2_1.png}
\end{figure}

\subsubsection{Antichains} 
\begin{enumerate}
\item $\{2,3\}$
\item $\{4,6\}$
\item $\{8,12\}$
\end{enumerate}
\subsubsection{Order Ideals}
\begin{enumerate}
\item $\{1,2,3\}$
\item $\{1,2,3,4,6\}$
\item $\{1,2,3,4,6,8,12\}$
\end{enumerate}
\subsubsection{Order Filters}
\begin{enumerate}
\item $\{2,3,4,6,8,12\}$
\item $\{4,6,8,12\}$
\item $\{8,12\}$
\end{enumerate}

\subsection{Problem 3: On Order Dimension (2 points)}

Given: a set $S$ with a relation $R$:

\begin{align*}
S &= \{1,2,3,5,6,10,15,30\} \\
R &= \{ a \leq b\ |\ b mod\ a = 0
\end{align*}

\subsubsection{Hasse diagram}
\begin{figure}[h]
\centering
\includegraphics[width=0.3\textwidth]{2_3_1.png}
\end{figure}

\subsubsection{Order dimension}

\begin{enumerate}
\item As we know, order dimension of the poset (R,S) is the minimal number of linear orders, constructed from the set S, which interception gives the considered poset.
\item Let the 1st linear order be the following topological sorting:
\[ S_1 = \{1,2,3,6,5,10,15,30\} \]
There are several connections which must be excluded:
\begin{align*}
&(2,3),(2,5),(2,15)\\
&(3,5),(3,10) \\
&(6,5),(6,10),(6,15)\\
&(10,15)
\end{align*}
\item Let the 2nd linear order be the another sorting:
\[S_2 = \{1,5,3,15,2,10,6,30\}\]
The only remained prohibited connecting is $(3,10)$.
\item Let the last linear order be the another sorting:
\[S_2 = \{1,2,5,10,3,6,15,30\}\]
There are no prohibited connections remained, so the interception of these 3 linear order will give us required poset.
\item \textbf{Answer:} the order dimension of the poset (R,S) is 3.
\end{enumerate}

\subsection{Problem 4: Big homework (2 points)}

The classification on the data selected in the previous homework was performed using the following 4 methods:

\begin{enumerate}
\item Decision tree
\item Random forest
\item xGboost
\item k-NN classifier
\end{enumerate}

The results presented on the figure \ref{fig:baseline}. 
It is seen that every model performs completely different depending on selected data.
The notebooks of binarization and classification are presented \href{https:/github.com/Chpel/Maga/tree/main/OSDA\%20.ipynb/Big\%20homework}{here}.

\begin{figure}[h]
\centering
\includegraphics[width=0.8\textwidth]{2_4_1.png}
\caption{Results of the classification research: number of datasets are the following: 1 - bikes, 2 - cancer, 3 - cars.}
\label{fig:baseline}
\end{figure}
